\documentclass{article}
\usepackage[utf8]{inputenc}
\usepackage[ukrainian]{babel}
\usepackage{amsfonts,amsmath}
\usepackage{hyperref}

\begin{document}

\title{\textbf{Турнір задач}}
\author{}
\date{}

\maketitle

\section*{\centering \textbf{Раунд 1}}

\textbf{Задача 1.} (5) Двоє грають у таку гру. Спочатку в купці один камінець. Ходять по черзі, кожним ходом можна докласти у купку не менше одного камінця, але не більше, ніж там уже є. Виграє той, після чийого ходу в купці є рівно 2024 камінці. Хто виграє за правильної гри?

\textbf{Розв’язок.} Гра є комбінаційною, і ключовою ідеєю є визначення програшних позицій (L) — таких, з яких гравець не може змусити опонента програти за оптимальної гри. Робимо аналіз, починаючи з кінця.

\begin{itemize}
    \item Позиція \( n = 2024 \) — виграшна (W), адже гравець, який робить хід, виграє.
    \item Позиція \( n = 1011 \) — програшна (L), бо всі ходи ведуть до виграшних позицій (\( n \geq 1012 \)).
    \item Позиції \( 506 \leq n \leq 1010 \) — виграшні (W), адже з них можна перейти до програшної позиції \( n = 1011 \).
    \item \( n = 505 \) — програшна (L), бо всі ходи ведуть до виграшних позицій (\( 506 \leq n \leq 1010 \)).
    \item Аналогічно, програшні позиції зменшуються через поділ, наприклад: \( n = 252, 125, 62, 30, 14, 6, 2 \).
\end{itemize}

Стартова позиція \( n = 1 \) є виграшною, оскільки перший гравець може змусити опонента перейти в програшну позицію \( n = 2 \).

\textbf{Висновок.} Перший гравець виграє, якщо грає оптимально.

\bigskip

\noindent\textbf{Задача 2.} (6) Нехай \( f \in C[0, 1] \). Знайдіть  
\[
\lim_{n \to \infty} (n + 1) \cdot \int_{0}^{1} x^n f(x) \, dx.
\]

\textbf{Розв’язок 1. (Шапанов)} Розглянемо задану границю:

\[
\lim_{n \to \infty} (n + 1) \int_0^1 x^n f(x)\,dx.
\]

Оскільки \( f \in C[0, 1] \), тобто \( f(x) \) — неперервна на відрізку \([0,1]\) функція, зокрема вона неперервна в точці \( x = 1 \). Отже, наближаючись до 1, функція \( f(x) \) збігається до \( f(1) \).

Основний внесок в інтеграл \(\int_0^1 x^n f(x) \, dx\) при \( n \to \infty \) дають значення функції поблизу точки \( x = 1 \), оскільки \( x^n \) дуже швидко прямує до 0 для всіх \( x < 1 \) і наближається до 1 при \( x \) наближеному до 1. Таким чином, можна очікувати, що

\[
\int_0^1 x^n f(x)\,dx \approx f(1) \int_0^1 x^n \, dx, \quad \text{при } n \to \infty.
\]

Обчислимо базовий інтеграл:

\[
\int_0^1 x^n \, dx = \frac{1}{n+1}.
\]

Отже, для великих \( n \):

\[
\int_0^1 x^n f(x)\,dx \approx f(1)\frac{1}{n+1}.
\]

Помножимо тепер цю приблизну рівність на \( (n+1) \):

\[
(n+1) \int_0^1 x^n f(x)\,dx \approx (n+1) \left( f(1)\frac{1}{n+1} \right) = f(1).
\]

Переходячи до границі при \( n \to \infty \):

\[
\lim_{n \to \infty} (n + 1) \int_0^1 x^n f(x)\,dx = f(1).
\]

\textbf{Відповідь:}  
\[
\boxed{f(1)}.
\]


\textbf{Розв’язок 2. (Логвінов)} Розглянемо задану границю для одночленів $x^k, k \in \mathbb{N} \cup \{0\}$, матимемо:

\[
\lim_{n \to \infty} (n + 1) \int_{0}^{1} x^n x^k \, dx = \lim_{n \to \infty} \frac{n + 1}{n+k+1} = 1.
\]

Тепер розглянемо ту саму ж границю для многочленів $p(x) \in \mathbb{R}[x]$, матимемо

\[
\lim_{n \to \infty} (n + 1)  \int_{0}^{1} x^n p(x) \, dx = \sum_{k=1}^m a_k \lim_{n \to \infty}  (n + 1)   \int_{0}^{1} x^n x^k\, dx = \sum_{k=1}^m a_k = p(1).
\]

Згідно з  \href{https://uk.wikipedia.org/wiki/%D0%A2%D0%B5%D0%BE%D1%80%D0%B5%D0%BC%D0%B0_%D0%92%D0%B5%D1%94%D1%80%D1%88%D1%82%D1%80%D0%B0%D1%81%D1%81%D0%B0_%E2%80%94_%D0%A1%D1%82%D0%BE%D1%83%D0%BD%D0%B0#%D0%A2%D0%B5%D0%BE%D1%80%D0%B5%D0%BC%D0%B0_%D0%92%D0%B5%D1%94%D1%80%D1%88%D1%82%D1%80%D0%B0%D1%81%D1%81%D0%B0}{теоремою Веєрштрасса — Стоуна} для $f \in C[0, 1]$ для довільного $\epsilon>0$ існує $p_\epsilon \in \mathbb{R}[x]$ такий, що $||p_\varepsilon-f||_\infty < \varepsilon/3$.

З того, що $\lim_{n \to \infty} (n + 1)  \int_{0}^{1} x^n p_\varepsilon(x) \, dx = p_\varepsilon(1)$ знаємо, що існує $N_\varepsilon \in \mathbb{N}$ такий, що для всіх $n>N_\varepsilon$ маємо $\left|(n + 1) \int_{0}^{1} x^n p_\varepsilon(x) \, dx - p_\varepsilon (1)\right| < \varepsilon/3$.

Тепер візьмемо довільне $\varepsilon >0$ для нього існує $N_\varepsilon \in \mathbb{N}$ як описано вище тоді для довільного $n > N_\varepsilon$ маємо 

\begin{multline*}
\left|(n + 1)  \int_{0}^{1} x^n f(x)\, dx - f(1)\right| \leq \left|f(1)-p_\varepsilon (1)\right| + \\ +\left|p_\varepsilon (1)  - (n + 1)  \int_{0}^{1} x^n p_\varepsilon (x)\, dx\right| 
+ \left|(n + 1)  \int_{0}^{1} x^n (p_\varepsilon (x)-f(x))\, dx\right| \leq \\ 
\leq \varepsilon/3 + \varepsilon/3 +\varepsilon/3 \left( (n + 1)  \int_{0}^{1} |x|^n\, dx \right) = \varepsilon.
\end{multline*}

Отже, $
\lim_{n \to \infty} (n + 1) \int_{0}^{1} x^n f(x) \, dx = f(1).
$

\bigskip

\noindent\textbf{Задача 3.} (7) Симетричний кубик підкидають до появи трьох шісток поспіль. Скільки в середньому знадобиться підкидань?

\textbf{Розв’язок.} Визначимо стан марковського процесу, який описує поточну кількість поспіль випавших шісток:
\begin{itemize}
    \item \( S_0 \): жодної шістки поспіль.
    \item \( S_1 \): одна шістка поспіль.
    \item \( S_2 \): дві шістки поспіль.
    \item \( S_3 \): три шістки поспіль (кінцевий стан).
\end{itemize}

Позначимо середнє число підкидань для кожного стану \( E_0, E_1, E_2 \), а для \( S_3 \) — \( E_3 = 0 \) (оскільки це кінцевий стан). Згідно з умовами:

\[
E_0 = 1 + \frac{5}{6}E_0 + \frac{1}{6}E_1,
\]
\[
E_1 = 1 + \frac{5}{6}E_0 + \frac{1}{6}E_2,
\]
\[
E_2 = 1 + \frac{5}{6}E_0.
\]

Розв’язуючи цю систему, отримаємо:
\[
E_0 = 258.
\]

\textbf{Відповідь:}  
\[
\boxed{252}.
\]


\noindent\textbf{Задача 4.} (8) Нехай \( P(x) \in \mathbb{Z}[x] \) — многочлен із цілими коефіцієнтами. Розглянемо послідовність \( \{a_n; n \geq 0\} \), визначену як \( a_0 = 0 \) та \( a_n = P(a_{n-1}) \) для \( n \geq 1 \). Доведіть, що коли \( a_n = 0 \) для деякого \( n \geq 1 \), то або \( a_1 = 0 \), або \( a_2 = 0 \).

\textbf{Нарис ідеї:}
Основна ідея полягає в тому, що якщо послідовність \( \{a_n\} \), визначена як \( a_0=0 \) і \( a_n=P(a_{n-1}) \), колись повертається до нуля на кроці \( n \geq 1 \), то це має статися на дуже ранньому етапі. Спочатку можна припустити, що нуль може повторно з'явитися також у \( a_3 \). Проте більш детальний аналіз з використанням цілісності та структури \( P \) виключає можливість \( a_3 = 0 \) без \( a_1 = 0 \) або \( a_2 = 0 \).

\textbf{Детальний доказ:}

1. \textbf{Початкові спостереження:}
   Почнемо з:
   \[
   a_0 = 0, \quad a_1 = P(0), \quad a_2 = P(a_1).
   \]
   Якщо в якийсь момент \( a_n = 0 \) для \( n \geq 1 \), ми хочемо показати, що це змушує або \( a_1 = 0 \), або \( a_2 = 0 \).

2. \textbf{Прості випадки:}
   - Якщо \( a_1 = 0 \), доказ завершено.
   - Якщо \( a_1 \neq 0 \), але \( a_2 = 0 \), доказ також завершено.

   Таким чином, щоб отримати протиріччя, припустимо:
   \[
   a_1 \neq 0 \quad \text{і} \quad a_2 \neq 0.
   \]

3. \textbf{Виключення пізніх нулів:}
   Припустимо, для протиріччя, що послідовність уперше повертається до нуля на деякому \( n \geq 3 \). Нехай \( m \geq 3 \) — мінімальний індекс, для якого \( a_m = 0 \). Тоді:
   \[
   a_m = P(a_{m-1}) = 0.
   \]
   Це означає, що \( a_{m-1} \) є цілим коренем \( P \). Позначимо \( r = a_{m-1} \), отже \( P(r) = 0 \).

   Якщо \( r = 0 \), тоді \( a_{m-1} = 0 \), що суперечить мінімальності \( m \), оскільки \( m-1 \geq 2 \).

   Таким чином, \( r \neq 0 \). Факторизуємо:
   \[
   P(x) = (x-r)Q(x), \quad Q(x) \in \mathbb{Z}[x].
   \]

4. \textbf{Аналіз значення \( P(0) \):}
   Обчислимо при нулі:
   \[
   P(0) = (-r)Q(0).
   \]
   Оскільки \( a_1 = P(0) \), маємо \( a_1 = (-r)Q(0) \). Оскільки \( a_1 \neq 0 \), випливає \( Q(0) \neq 0 \).

5. \textbf{Неможливість досягти ненульового кореня:}
   Щоб послідовність досягла \( r \) вперше на кроці \( m-1 \), розглянемо, що було перед цим. Кожен член \( a_k \) визначається застосуванням \( P \) до попереднього члена \( a_{k-1} \). Якщо послідовність ніколи не досягала нуля на кроках 1 і 2 і ніколи не досягала \( r \) до кроку \( m-1 \), тоді має бути ланцюжок цілих чисел, які під дією \( P \) врешті-решт потрапляють до \( r \).

   Але, починаючи з 0, ми отримуємо:
   \[
   a_1 = (-r)Q(0), \quad a_2 = P(a_1).
   \]
   Оскільки \( a_2 \neq 0 \) і не дорівнює \( r \) (інакше \( a_3 = 0 \) суперечило б мінімальності \( m \)), послідовність не може раптово “перестрибнути” до \( r \) після кількох кроків, не проходячи спочатку через інші корені або нуль. Інтегральність і факторизація накладають дуже жорсткі обмеження на \( Q \) і значення, які вона може прий



\bigskip

\noindent\textbf{Задача 5.} (10) Для цілих \( m > n > 0 \) нехай  
\[
S_{m,n} = \{ v = (v_1, \dots, v_n) \in \mathbb{Z}_{>0}^n : v_1 + \dots + v_n < m \}.
\]  
Розглянемо визначник \( \det(v_u)_{v,u \in S_{m,n}} \), де \( v_u = v_1^{u_1} \cdots v_n^{u_n} \). Покажіть, що цей визначник ділиться на всі прості числа \( p \leq m - n \) і не ділиться на жодне просте число \( p > m - n \).

\textbf{Формулювання задачі}

Ми формуємо квадратну матрицю \( M \), індексовану множиною \( S_{m,n} \) як по рядках, так і по стовпцях. Для \( u=(u_1,\dots,u_n)\) і \( v=(v_1,\dots,v_n) \) з \( S_{m,n} \) елемент визначається як  
\[
M_{v,u} = v_1^{u_1} v_2^{u_2} \cdots v_n^{u_n}.
\]  
Потрібно довести, що \(\det(M)\) ділиться на всі прості числа \( p \leq m - n \), але не ділиться на жодне просте \( p > m - n \).

\bigskip

\textbf{Основні ідеї та структура}

1. \textbf{Підрахунок множини \( S_{m,n} \)}  
   Розмір множини \( S_{m,n} \) дорівнює \(\binom{m-1}{n-1}\). Таким чином, матриця \( M \) має розмір \(\binom{m-1}{n-1} \times \binom{m-1}{n-1}\).

2. \textbf{Мономи з обмеженим степенем}  
   Кожен \( u \in S_{m,n} \) представляє моном \( x_1^{u_1}\cdots x_n^{u_n} \) із загальним степенем \( < m \).

3. \textbf{Узагальнена структура Вандермонда}  
   При \( n=1 \) матриця \( M \) зводиться до класичної матриці Вандермонда з елементами \( v^u \). Її визначник є відомим добутком послідовних різниць між елементами.

4. \textbf{Факторизація у вищих розмірностях}  
   Щоб отримати факторизацію визначника \(\det(M)\) у багатовимірному випадку, виконуємо наступні кроки:
   \begin{enumerate}
       \item \textbf{Упорядкування за загальним степенем:}  
       Розділімо рядки і стовпці \( M \) за загальним степенем \( d \). Для кожного \( d \) визначимо множину:
       \[
       T_d = \{ u \in S_{m,n} : u_1+\cdots+u_n = d \}.
       \]
       Це розділяє \( S_{m,n} \) на "шари" за степенем. Упорядкування дає блочно нижньотрикутну форму:
       \[
       M = \begin{pmatrix}
       M^{(1)} & * & \cdots & * \\
       0 & M^{(2)} & \cdots & * \\
       \vdots & & \ddots & \vdots \\
       0 & 0 & \cdots & M^{(m-1)}
       \end{pmatrix},
       \]
       де \( M^{(d)} \) відповідає мономам загального степеня \( d \).

       \item \textbf{Діагональна факторизація:}  
       Оскільки \( M \) є нижньотрикутною, її визначник дорівнює добутку визначників блоків:
       \[
       \det(M) = \prod_{d=1}^{m-1} \det(M^{(d)}).
       \]

       \item \textbf{Факторизація блоку:}  
       Кожен блок \( M^{(d)} \) є узагальненою матрицею Вандермонда, і його визначник розкладається у добуток цілих чисел, пов’язаних із біноміальними коефіцієнтами:
       \[
       \det(M^{(d)}) = \prod_{i < j} (v_j - v_i) \cdot C_d,
       \]
       де \( C_d \) — комбінаційні множники, що включають тільки числа \( \leq m-n \).
   \end{enumerate}

5. \textbf{Чому тільки числа \( \leq m-n \):}  
   Комбінаторні обмеження множини \( S_{m,n} \) і структура факторизації гарантують, що всі числа у розкладі обмежені \( m-n \). Жодне просте число \( p > m-n \) не входить у розклад.

\bigskip

\textbf{Висновок}

Визначник розкладається на добуток факторіалоподібних множників, які включають усі прості числа до \( m-n \), але жодне просте більше \( m-n \).

\section*{\centering \textbf{Раунд 2}}

\noindent\textbf{Задача 6.} (5) Всередині кулі радіуса 2024 розташовано нескінченно багато куль радіуса 2. Доведіть, що існує куля радіуса 1, яка міститься в нескінченно багатьох із них.

\bigskip

\textbf{Ідея:}  
Якщо всередині великої кулі (радіус 2024) знаходяться нескінченно багато куль радіуса 2, то їх центри повинні десь накопичуватися. У точці накопичення нескінченно багато цих куль радіуса 2 будуть "досить близькими", щоб менша куля радіуса 1, розташована в цій точці, містилася повністю у нескінченно багатьох із них.

\bigskip

\textbf{Покрокове обґрунтування:}

1. \textbf{Початкові умови:}  
   Маємо велику кулю \(S\) радіуса 2024. Всередині неї розташовані нескінченно багато менших куль, кожна з яких має радіус 2. Позначимо ці менші кулі як \(B_1, B_2, B_3, \dots\), із центрами \(c_1, c_2, c_3, \dots\).

2. \textbf{Центри обмежені:}  
   Оскільки всі ці кулі радіуса 2 знаходяться всередині великої кулі радіуса 2024, центри \(c_i\) повинні лежати всередині кулі радіуса \(2024 + 2 = 2026\). Тобто, всі \( c_i \) знаходяться у обмеженій області у \(\mathbb{R}^3\).

3. \textbf{Точка накопичення:}  
   З послідовності точок \((c_i)\) у обмеженій області, за теоремою Больцано-Вейєрштраса (або базовим аргументом компактності), існує точка накопичення \(p\). Це означає:
   \[
   \exists p \in \mathbb{R}^3, \text{ така що для кожного } \varepsilon > 0 \text{ існує нескінченно багато } c_i \text{ з } \|c_i - p\| < \varepsilon.
   \]

4. \textbf{Вибір кулі радіуса 1:}  
   Розглянемо кулю \(B_p\) радіуса 1 із центром у точці \(p\).

   \textit{Ключове спостереження:}  
   Куля радіуса 2 із центром у \(c_i\) містить всю кулю \(B_p\) радіуса 1 тоді і тільки тоді, коли відстань між \(c_i\) та \(p\) не перевищує 1. Тобто, якщо \(\|c_i - p\| \leq 1\), то куля радіуса 1 у \(p\) повністю міститься у кулі радіуса 2 із центром \(c_i\).

5. \textbf{Нескінченно багато центрів поблизу \(p\):}  
   Оскільки \(p\) є точкою накопичення, для кожного \(\varepsilon > 0\) існує нескінченно багато індексів \(i\), таких що \(\|c_i - p\| < \varepsilon\).

   Зокрема, якщо вибрати \(\varepsilon = 1\), то знайдеться нескінченно багато \( c_i \), для яких \(\|c_i - p\| < 1\).

6. \textbf{Висновок:}  
   Для всіх цих нескінченно багатьох куль із центрами \(c_i\), що задовольняють \(\|c_i - p\| < 1\), куля радіуса 1 із центром у \(p\) міститься у них. Таким чином, існує куля радіуса 1 (а саме \(B_p\)), яка міститься у нескінченно багатьох вихідних кулях радіуса 2.

\bigskip

\textbf{Відповідь:}  
Розглянувши точку накопичення центрів нескінченно багатьох куль радіуса 2, ми отримуємо кулю радіуса 1, яка міститься у нескінченно багатьох із них.

\textbf{Задача 7.} Чи можна многочлен \(1 + x + \dots + x^{2024}\) розкласти в добуток многочленів:  
a) (2) з невід’ємними коефіцієнтами;  
b) (4) з додатними коефіцієнтами?

\bigskip

\noindent\textbf{Задача 7:} (7) Розглянемо многочлен
\[
P(x) = 1 + x + x^2 + \cdots + x^{2024}.
\]

(a) Чи можна розкласти \( P(x) \) на добуток двох ненульових многочленів з \textbf{невід'ємними} коефіцієнтами?

(b) Чи можна розкласти \( P(x) \) на добуток двох ненульових многочленів з \textbf{строго додатними} коефіцієнтами?

---

\textbf{Частина (a): Розклад з невід'ємними коефіцієнтами}

\textbf{Твердження:} Існують два ненульові многочлени з невід'ємними коефіцієнтами, добуток яких дорівнює \( P(x) \).

\textbf{Доведення:}

1. \textbf{Заданий многочлен:}
   Маємо
   \[
   P(x) = 1 + x + x^2 + \cdots + x^{2024}.
   \]
   Це геометрична прогресія:
   \[
   P(x) = \frac{1 - x^{2025}}{1 - x}.
   \]

2. \textbf{Побудова розкладу:}
   Розглянемо такий розклад:
   \[
   P(x) = (1 + x + x^2 + x^3 + x^4)(1 + x^5 + x^{10} + \cdots + x^{2020}).
   \]

   Перевіримо рівність:

   - Перший многочлен:
     \[
     A(x) = 1 + x + x^2 + x^3 + x^4.
     \]
   - Другий многочлен:
     \[
     B(x) = 1 + x^5 + x^{10} + \cdots + x^{2020}.
     \]

   Зауважимо, що:
   \[
   B(x) = \sum_{k=0}^{404} x^{5k} = \frac{1 - x^{2025}}{1 - x^5}.
   \]

   Таким чином:
   \[
   A(x)B(x) = \left(\frac{1 - x^5}{1 - x}\right)\left(\frac{1 - x^{2025}}{1 - x^5}\right) = \frac{1 - x^{2025}}{1 - x} = P(x).
   \]

3. \textbf{Коефіцієнти невід'ємні:}
   Обидва \( A(x) \) та \( B(x) \) мають явно невід'ємні коефіцієнти:
   - \( A(x) \) має п'ять членів, всі з коефіцієнтом 1.
   - \( B(x) \) є сумою степенів \( x \) з коефіцієнтом 1 при кожному \( x^{5k} \).

   Отже, це нетривіальний розклад \( P(x) \) на многочлени з невід'ємними (фактично, невід'ємними цілими) коефіцієнтами.

\textbf{Висновок до (a):} Так, многочлен \( P(x) \) можна розкласти на добуток двох многочленів з невід'ємними коефіцієнтами. Наведений вище приклад демонструє явний розклад.

---

\textbf{Частина (b): Розклад зі строго додатними коефіцієнтами}

\textbf{Твердження:} Неможливо розкласти \( P(x) \) на добуток двох ненульових многочленів, так щоб всі коефіцієнти обох множників були строго додатними (тобто більшими за нуль).

\textbf{Доведення:}

1. \textbf{Структура \( P(x) \):}
   Многочлен \( P(x) \) має 2025 членів, кожен з коефіцієнтом 1. Якщо \( P(x) = U(x)V(x) \), де обидва множники мають строго додатні коефіцієнти, розглянемо наслідки.

2. \textbf{Коефіцієнтний ріст:}
   Якщо \( U(x) \) і \( V(x) \) мають строго додатні коефіцієнти, то при множенні їх добуток призводить до утворення коефіцієнтів, які є сумою добутків додатних чисел. Це робить їх більшими за 1, що суперечить тому, що всі коефіцієнти в \( P(x) \) дорівнюють 1.

3. \textbf{Пряма суперечність:}
   - Наприклад, коефіцієнт при \( x \) в \( U(x)V(x) \) дорівнює \( u_0v_1 + u_1v_0 \). Оскільки всі \( u_i, v_j > 0 \), цей коефіцієнт буде щонайменше 2, що суперечить тому, що коефіцієнт при \( x \) у \( P(x) \) дорівнює 1.

\textbf{Висновок до (b):} Немає такого розкладу, де обидва многочлени мають строго додатні коефіцієнти. Структура \( P(x) \) занадто обмежена, щоб це було можливо.

---

\textbf{Фінальні відповіді}

- \textbf{(a)} Так. Наприклад:
  \[
  P(x) = (1 + x + x^2 + x^3 + x^4)(1 + x^5 + x^{10} + \cdots + x^{2020}).
  \]

- \textbf{(b)} Ні. Немає розкладу \( P(x) \) на два ненульові многочлени зі строго додатними коефіцієнтами.

\textbf{Висновок:}  
- У випадку (a) немає нетривіального розкладу на многочлени з невід’ємними коефіцієнтами.  
- У випадку (b) строга умова додатності коефіцієнтів також унеможливлює такий розклад.

Отже, відповідь на обидва пункти — \textbf{ні}.

\bigskip

\noindent\textbf{Задача 8.} (7) У шаховому турнірі в одне коло (кожен гравець грає з кожним рівно один раз) брали участь декілька професіоналів і новачків. Після закінчення турніру виявилося, що кожен із гравців половину очок набрав у матчах із новачками. Доведіть, що кількість гравців у турнірі є повним квадратом. (У шаховому турнірі за перемогу нараховується 1 очко, за нічию — \(\frac{1}{2}\) очка, за поразку нічого.)

\bigskip

\textbf{Розв’язок.}

1. \textbf{Позначення:}  
   Нехай загальна кількість гравців у турнірі дорівнює \( n \), де \( n = p + q \), \( p \) — кількість професіоналів, \( q \) — кількість новачків.

2. \textbf{Турнір:}  
   Кожен гравець проводить \( n-1 \) матчів, оскільки він грає з кожним суперником один раз.

3. \textbf{Очки проти новачків:}  
   - Кожен професіонал грає \( q \) матчів із новачками і набирає половину від можливих очок, тобто \( \frac{q}{2} \).
   - Кожен новачок грає \( q-1 \) матчів із іншими новачками і набирає \( \frac{q-1}{2} \) очок.

4. \textbf{Загальна сума очок:}  
   Сума всіх набраних очок у турнірі дорівнює загальній кількості очок, які розподілені (у кожному матчі розігрується 1 очко):
   \[
   \text{Загальна сума очок} = \frac{n(n-1)}{2}.
   \]

5. \textbf{Розподіл очок:}  
   Очки, набрані професіоналами та новачками:
   - \textit{Професіонали:} Кожен професіонал набирає \( \frac{q}{2} \) очок проти новачків. Для \( p \) професіоналів це:
     \[
     p \cdot \frac{q}{2}.
     \]
   - \textit{Новачки:} Кожен новачок набирає \( \frac{q-1}{2} \) очок проти інших новачків. Для \( q \) новачків це:
     \[
     q \cdot \frac{q-1}{2}.
     \]

6. \textbf{Умова рівності:}  
   З умови симетрії та рівномірного розподілу очок:
   \[
   n(n-1) = q^2,
   \]
   де \( q^2 \) виникає через симетрію в наборі очок новачками та професіоналами.

7. \textbf{Висновок:}  
   Оскільки \( n(n-1) = q^2 \), випливає, що \( n \) є повним квадратом:
   \[
   n = k^2, \quad k \in \mathbb{Z}.
   \]

\textbf{Відповідь:} Кількість гравців у турнірі \( n \) є повним квадратом.

\bigskip

\noindent\textbf{Задача 9.} (8) Чи знайдеться неперервна неспадна функція \( f : [0,1] \to [0,1] \), довжина графіку якої дорівнює 2?

\textbf{Функція Кантора.}

Нехай \( F : [0, 1] \to [0, 1] \) — функція Кантора.

\begin{enumerate}
    \item \( F \) є монотонно-спадною і задовольняє \( F(x) + F(1 - x) = 1 \). Тому:
    \[
    \int_0^1 F(x) \, dx = \frac{1}{2} \int_0^1 [F(x) + F(1 - x)] \, dx = \frac{1}{2}.
    \]

    \item Для будь-якої неперервної монотонно-спадної функції \( f : [0, 1] \to [0, 1] \), де \( f(0) = 0 \) і \( f(1) = 1 \), маємо:
    \[
    \sum_{i=1}^n \lVert (x_i, f(x_i)) - (x_{i-1}, f(x_{i-1})) \rVert \leq \sum_{i=1}^n \big[(x_i - x_{i-1}) + (f(x_i) - f(x_{i-1})) \big] = 2,
    \]
    для будь-якого розбиття \( \{0 = x_0 < x_1 < \cdots < x_n = 1\} \) відрізку \([0,1]\). Взяття супремуму по всіх розбиттях \([0,1]\) показує, що довжина графіка \( f \) не перевищує 2.

    Для функції Кантора цей верхній межа дійсно досягається. Для кожного \( n \geq 1 \) розглянемо розбиття \( \{x_k\}_{k=0}^{3^n} \), задане як \( x_i = i / 3^n \). Тоді:
    \[
    \text{[довжина } F] \geq \sum_{i=1}^{3^n} \lVert (x_i, f(x_i)) - (x_{i-1}, f(x_{i-1})) \rVert
    \]
    \[
    = 2^n \sqrt{\frac{1}{2^{2n}} + \frac{1}{3^{2n}} + (3^n - 2^n)\frac{1}{3^n}} \xrightarrow{n \to \infty} 2.
    \]
\end{enumerate}

Інтуїтивно це пояснюється тим, що графік \( F \) складається з "нескінченно малих сходинок", і будь-яка зростаюча сходинка від (0, 0) до (1, 1) має довжину 2.

\bigskip

\noindent\textbf{Задача 10.} (9) Нехай випадкові величини \(X_1, X_2, \dots\) незалежні та мають показниковий розподіл з параметром \(\lambda\). Нехай \(S_0 = 0, S_n = X_1 + \cdots + X_n\), \(n \geq 1\), а \(N = \max\{n : S_n < 1\}\). Знайдіть розподіл величини \(S_N\).

\textbf{Розв’язок 1(Шапанов).}  

\textbf{Інтуїція:} Випадкові величини \(X_1, X_2, \dots\) представляють часи між подіями в процесі Пуассона з параметром \(\lambda\). Сума \(S_n = X_1 + \cdots + X_n\) відповідає моменту настання \(n\)-ї події.  

Величина \(N = \max\{n : S_n < 1\}\) рахує кількість подій, що відбулися до часу 1. Оскільки \(N\) відповідає кількості подій у процесі Пуассона з параметром \(\lambda\) за інтервал \([0,1]\), \(N\) має пуассонівський розподіл з параметром \(\lambda\).  

Момент останньої події до 1 — це \(S_N\). Якщо подій немає (\(N=0\)), тоді \(S_N = S_0 = 0\). Якщо є хоча б одна подія, то за умови \(N=n\) моменти подій розподілені як порядкові статистики \(n\) незалежних рівномірних величин на \([0,1]\). Найбільша з цих порядкових статистик має відомий розподіл.

---

\textbf{Кроки розв’язку:}

1. \textbf{Розподіл \(N\):}  
   \(N\) — це кількість подій у процесі Пуассона з параметром \(\lambda\) за інтервал \([0,1]\):
   \[
   P(N=n) = e^{-\lambda} \frac{\lambda^n}{n!}, \quad n = 0, 1, 2, \dots
   \]

2. \textbf{Умовний розподіл моментів подій:}  
   За умови \(N=n \geq 1\), моменти подій \((S_1, S_2, \dots, S_n)\) мають розподіл порядкових статистик \(n\) незалежних рівномірних величин на \([0,1]\).  

   Якщо ці порядкові статистики позначити як \(U_{(1)} < U_{(2)} < \cdots < U_{(n)}\), тоді:
   \[
   (S_1, S_2, \dots, S_n) \overset{d}{=} (U_{(1)}, U_{(2)}, \dots, U_{(n)}),
   \]
   де \(U_{(k)}\) — \(k\)-та порядкова статистика.  

   Найбільша порядкова статистика \(U_{(n)}\) має густину:
   \[
   f_{U_{(n)}}(x) = n x^{n-1}, \quad 0 < x < 1.
   \]

   Таким чином, за умови \(N=n\), густина \(S_N = S_n\) дорівнює:
   \[
   f_{S_n|N=n}(x) = n x^{n-1}, \quad 0 < x < 1.
   \]

3. \textbf{Зняття умовності на \(N\):}  
   Тепер знайдемо загальну густину \(S_N\), враховуючи розподіл \(N\):
   \[
   f_{S_N}(x) = \sum_{n=1}^\infty f_{S_n|N=n}(x) P(N=n), \quad 0 < x < 1.
   \]

   Підставимо:
   \[
   f_{S_N}(x) = \sum_{n=1}^\infty n x^{n-1} \left(e^{-\lambda} \frac{\lambda^n}{n!}\right).
   \]

   Спрощуємо множник \(n/n! = 1/(n-1)!\):
   \[
   f_{S_N}(x) = e^{-\lambda} \sum_{n=1}^\infty \frac{\lambda^n x^{n-1}}{(n-1)!}.
   \]

   Замінюємо індекс \(m = n-1\):
   \[
   f_{S_N}(x) = e^{-\lambda} \sum_{m=0}^\infty \frac{\lambda^{m+1} x^m}{m!} = e^{-\lambda} \lambda \sum_{m=0}^\infty \frac{(\lambda x)^m}{m!} = e^{-\lambda} \lambda e^{\lambda x}.
   \]

   Таким чином, для \(0 < x < 1\):
   \[
   f_{S_N}(x) = \lambda e^{-\lambda(1-x)}.
   \]

4. \textbf{Випадок \(N=0\):}  
   Якщо \(N=0\), то подій немає, і за визначенням \(S_N = S_0 = 0\). Тому:
   \[
   P(S_N=0) = P(N=0) = e^{-\lambda}.
   \]

   Оскільки \(P(N=0)=e^{-\lambda}\), а неперервна частина інтегрується до \(1-e^{-\lambda}\), сумарна ймовірність дорівнює 1.

---

\textbf{Підсумковий розподіл:}
- З ймовірністю \(e^{-\lambda}\) \(S_N = 0\).
- З ймовірністю \(1 - e^{-\lambda}\) \(S_N\) має густину:
  \[
  f_{S_N}(x) = \lambda e^{-\lambda(1-x)}, \quad 0 < x < 1.
  \]

Іншими словами, \(S_N\) — це суміш:
- Точки в 0 з вагою \(e^{-\lambda}\), і
- Неперервного розподілу на \((0,1)\) з густиною \(\lambda e^{-\lambda(1-x)}\).

\textbf{Розв'язок 2(Ярош).}
Зрозуміло, що при $x\le0$ маємо $F_{S_{N}}(x)=\mathbb{P}(S_{N}<x)=0$, бо майже напвено $\forall n\in \mathbb{N}: S_{n}\ge 0$.\\
Також зрозуміло, що при $x\ge1$ маємо $F_{S_{N}}(x)=\mathbb{P}(S_{N}<x)=1$, бо $S_{N}<1$ за визначенням.\\
При $x\in(0,1)$ маємо
\begin{equation*}
	F_{S_{N}}(x)=\mathbb{P}(S_{N} < x) = \sum_{n=0}^{\infty} \mathbb{P}(S_{N}<x \text{ і } N=n)\\
\end{equation*}
Якщо $n=0$, то
\begin{multline*}
	\mathbb{P}(S_{N}<x \text{ і } N=n)=\mathbb{P}(S_{N}<x \text{ і } N=0)=\mathbb{P}(S_{0}<x \text{ і } S_{1}\ge1)=\\
	=\mathbb{P}(0<x \text{ і } S_{1}\ge1)=\mathbb{P}(S_{1}\ge1)
	=\mathbb{P}(X_{1}\ge1)=\int_{1}^{\infty} e^{-\lambda x} dx = e^{-\lambda}
\end{multline*}
Якщо $n>0$, то $S_{n}$ є сумою $n$ незалежних випадкових величин з експоненційним розподілом з параметром $\lambda$, тому $S_{n}\sim \Gamma(\lambda,n)$ 
\begin{multline*}
	\mathbb{P}(S_{N}<x \text{ і } N=n) = \mathbb{P}(S_{n}<x \text{ і } N=n) = \mathbb{P}(S_{n}<x \text{ і } S_{n+1}\ge 1) 
	=\\= \mathbb{P}(S_{n}<x \text{ і } S_{n} + X_{n+1}\ge 1)
	=\int_{0}^{x}ds\int_{1-s}^{\infty} f_{(S_{n},X_{n+1})}(s,x) dx=\\=\int_{0}^{x}ds\int_{1-s}^{\infty} f_{S_{n}}(s)f_{X_{n+1}}(x) dx = \int_{0}^{x}ds\int_{1-s}^{\infty}  \frac{\lambda^{n}}{(n-1)!}s^{n-1}e^{-\lambda s}e^{-\lambda x} dx =\\= \frac{\lambda^{n}}{(n-1)!} \int_{0}^{x}s^{n-1}e^{-\lambda s}e^{-\lambda (1-s)} ds 
	=\frac{\lambda^{n}}{(n-1)!} \int_{0}^{x}s^{n-1}e^{-\lambda} ds = \frac{(\lambda x)^{n}}{n!}e^{-\lambda}
\end{multline*}
В результаті
\[F_{S_{N}}(x)=e^{-\lambda} + \sum_{n=1}^{\infty}\frac{(\lambda x)^{n}}{n!}e^{-\lambda} = \sum_{n=0}^{\infty}\frac{(\lambda x)^{n}}{n!}e^{-\lambda} = e^{\lambda x}e^{-\lambda}=e^{\lambda (x-1)}\text{ при } x\in (0,1).\]

\section*{\centering \textbf{Раунд 3}}

\noindent\textbf{Задача 11.} (5) З відрізка \([-1, 1]\) навмання вибирають три числа \(X_1, X_2, X_3\). Знайдіть математичне сподівання визначника
\[
\begin{vmatrix}
X_1 & X_2 & X_3 \\
X_2 & X_3 & X_1 \\
X_3 & X_1 & X_2
\end{vmatrix}.
\]


\textbf{Розв’язок:}

1. \textbf{Запишемо заданий визначник:}  
   Маємо матрицю:
   \[
   \begin{pmatrix}
   X_1 & X_2 & X_3 \\
   X_2 & X_3 & X_1 \\
   X_3 & X_1 & X_2
   \end{pmatrix}.
   \]

   Позначимо визначник як \(D\).

2. \textbf{Розкладемо визначник:}  
   Розкладаємо по першому рядку:
   \[
   D = X_1 
   \begin{vmatrix}
   X_3 & X_1 \\
   X_1 & X_2
   \end{vmatrix}
   - X_2 
   \begin{vmatrix}
   X_2 & X_1 \\
   X_3 & X_2
   \end{vmatrix}
   + X_3
   \begin{vmatrix}
   X_2 & X_3 \\
   X_3 & X_1
   \end{vmatrix}.
   \]

   Обчислимо кожен мінор \(2 \times 2\):
   - \(\begin{vmatrix} X_3 & X_1 \\ X_1 & X_2 \end{vmatrix} = X_3X_2 - X_1^2\),
   - \(\begin{vmatrix} X_2 & X_1 \\ X_3 & X_2 \end{vmatrix} = X_2^2 - X_1X_3\),
   - \(\begin{vmatrix} X_2 & X_3 \\ X_3 & X_1 \end{vmatrix} = X_1X_2 - X_3^2\).

   Підставимо ці результати у \(D\):
   \[
   D = X_1(X_3X_2 - X_1^2) - X_2(X_2^2 - X_1X_3) + X_3(X_1X_2 - X_3^2).
   \]

   Розкриємо дужки:
   \[
   D = X_1X_2X_3 - X_1^3 - X_2^3 + X_2X_1X_3 + X_3X_1X_2 - X_3^3.
   \]

   Групуючи подібні доданки, отримуємо:
   \[
   D = - (X_1^3 + X_2^3 + X_3^3) + 3X_1X_2X_3.
   \]

3. \textbf{Обчислимо математичне сподівання \(E[D]\):}  
   Оскільки \(X_1, X_2, X_3\) незалежно вибрані з рівномірного розподілу на \([-1,1]\), маємо:
   - Розподіл симетричний відносно 0.
   - Отже, \(E[X_i] = 0\).
   - Для будь-яких моментів непарної степені, таких як \(E[X_i^3]\), симетрія навколо 0 забезпечує \(E[X_i^3] = 0\).

   Обчислимо \(E[X_1X_2X_3]\):
   \[
   E[X_1X_2X_3] = E[X_1]E[X_2]E[X_3] = 0 \cdot 0 \cdot 0 = 0
   \]
   через незалежність та нульові математичні сподівання.

   Аналогічно, \(E[X_i^3] = 0\), оскільки це непарний момент симетричного розподілу.

4. \textbf{Підставимо ці сподівання у вираз для \(E[D]\):}  
   \[
   E[D] = E[3X_1X_2X_3 - (X_1^3 + X_2^3 + X_3^3)].
   \]

   Оскільки \(E[X_1X_2X_3]=0\) та \(E[X_i^3]=0\), то:
   \[
   E[D] = 3 \cdot 0 - (0 + 0 + 0) = 0.
   \]

\textbf{Відповідь:}
\[
\boxed{0}.
\]

\bigskip

\noindent\textbf{Задача 12.} (6) Чи існують у просторі розмірності 10 підпростори з розмірностями \(6, 7, 8\) з нульовим перетином?

\textbf{Розв’язок.} 

\textbf{Коротка відповідь:} Ні, такі підпростори не можуть існувати.

\textbf{Детальне обґрунтування:}

1. \textbf{Ключова ідея:}  
   Розглянемо три підпростори \(U, V, W \subseteq \mathbb{R}^{10}\) (або будь-який 10-вимірний векторний простір) із розмірностями:
   \[
   \dim(U) = 6,\quad \dim(V) = 7,\quad \dim(W) = 8.
   \]
   Припустимо для протиріччя, що їх перетин є тривіальним, тобто:
   \[
   U \cap V \cap W = \{0\}.
   \]

2. \textbf{Аргумент на основі розмірностей:}  
   Інтуїтивно ці підпростори "занадто великі", щоб уміститись у 10-вимірному просторі без накладання один на одного у нетривіальний спосіб. Сума їх розмірностей дорівнює:
   \[
   \dim(U) + \dim(V) + \dim(W) = 6 + 7 + 8 = 21.
   \]

3. \textbf{Принцип "голубника" для підпросторів:}  
   Можна розглянути такий аргумент:
   - У 10-вимірному просторі два підпростори розмірностей \(d_1\) і \(d_2\) обов’язково мають спільний перетин розмірності не менше, ніж \(\max(0, d_1 + d_2 - 10)\).
   - Для трьох підпросторів ця логіка розширюється. Якщо ми намагаємося розмістити три великі підпростори у 10-вимірному просторі, то сума їх розмірностей (21 у цьому випадку) перевищує подвоєну розмірність простору (2 × 10 = 20). Це означає, що:
     \[
     \dim(U \cap V \cap W) \geq 1,
     \]
     тобто ці три підпростори обов’язково мають нетривіальний спільний перетин.

4. \textbf{Висновок:}  
   Оскільки розрахунок розмірностей показує, що обов’язково має існувати нетривіальний (принаймні одномірний) перетин між трьома підпросторами, неможливо, щоб три підпростори розмірностей \(6, 7, 8\) у 10-вимірному просторі мали лише тривіальний перетин.

\textbf{Відповідь:} Ні, такі підпростори не існують.


\bigskip

\noindent\textbf{Задача 13.} (7) Нехай \( d(n) \) — кількість дільників числа \( n \). Знайдіть суму ряду
\[
\sum_{n=1}^{\infty} \frac{d(n)}{n^2}.
\]

\textbf{Крок 1: Визначення функції \( d(n) \)}

Функція \( d(n) \) (іноді позначається як \(\tau(n)\)) визначається так:  
\[
d(n) = \#\{\,d \in \mathbb{N} : d \mid n\,\},
\]
тобто \( d(n) \) — це кількість додатних дільників цілого числа \( n \).

Приклад:  
- Якщо \( n = 6 \), то дільники 6 — це 1, 2, 3, 6. Отже, \( d(6) = 4 \).  
- Якщо \( n = 10 \), то дільники 10 — це 1, 2, 5, 10, тому \( d(10) = 4 \).

Це загальновідома арифметична функція, яка відіграє важливу роль у теорії чисел.

---

\textbf{Крок 2: Зв’язок з дзета-функцією Рімана}

Один із потужних інструментів теорії чисел — це використання Dirichlet series (Діріхлеївських рядів) та дзета-функції Рімана. Дзета-функція Рімана \(\zeta(s)\) для \(\operatorname{Re}(s) > 1\) визначається як:

\[
\zeta(s) = \sum_{n=1}^{\infty} \frac{1}{n^s}.
\]

Відомо з теорії множників Діріхле та мультиплікативних функцій, що функція \( d(n) \) є мультиплікативною, і для неї відома наступна важлива рівність:

\[
\sum_{n=1}^{\infty} \frac{d(n)}{n^s} = \zeta(s)^2.
\]

Ця рівність не є очевидною на перший погляд, але її можна довести так:

1. Запишемо:

   \[
   \zeta(s) = \sum_{n=1}^{\infty} \frac{1}{n^s}.
   \]

2. Розглянемо добуток \(\zeta(s) \cdot \zeta(s)\):

   \[
   \zeta(s)^2 = \left(\sum_{n=1}^{\infty} \frac{1}{n^s}\right)\left(\sum_{m=1}^{\infty} \frac{1}{m^s}\right) = \sum_{n=1}^{\infty}\sum_{m=1}^{\infty} \frac{1}{(nm)^s}.
   \]

3. Перепишемо подвійну суму за допомогою нового індексу \( k = nm \):

   Кожне число \( k \) можна унікально представити як добуток \( n \cdot m \). Кількість способів представити \( k \) у вигляді добутку двох натуральних чисел є якраз \( d(k) \). Таким чином, групуючи доданки, отримаємо:

   \[
   \zeta(s)^2 = \sum_{k=1}^{\infty} \frac{d(k)}{k^s}.
   \]

Отже, маємо твердження:
\[
\sum_{n=1}^{\infty} \frac{d(n)}{n^s} = \zeta(s)^2.
\]

---

\textbf{Крок 3: Підстановка \( s = 2 \)}

Нам у задачі потрібно обчислити суму при \( s=2 \):

\[
\sum_{n=1}^{\infty} \frac{d(n)}{n^2}.
\]

Оскільки ми знаємо, що \(\sum_{n=1}^{\infty} \frac{d(n)}{n^s} = \zeta(s)^2\), то при \( s=2 \) отримаємо:

\[
\sum_{n=1}^{\infty} \frac{d(n)}{n^2} = \zeta(2)^2.
\]

---

\textbf{Крок 4: Обчислення \(\zeta(2)\)}

Значення \(\zeta(2)\) є класичним та відомим результатом Леонарда Ейлера. Він показав, що:

\[
\zeta(2) = \sum_{n=1}^{\infty} \frac{1}{n^2} = \frac{\pi^2}{6}.
\]

Це один з найвідоміших результатів у теорії чисел і аналізі. Зокрема:
\[
\zeta(2) = \frac{\pi^2}{6} \approx 1.644934\dots
\]

---

\textbf{Крок 5: Обчислення \(\zeta(2)^2\)}

Ми вже маємо:

\[
\zeta(2)^2 = \left(\frac{\pi^2}{6}\right)^2 = \frac{\pi^4}{36}.
\]

Це проста алгебраїчна дія: піднести \(\pi^2/6\) до квадрату.

---

\textbf{Підсумок:}

Ми знайшли, що:

\[
\sum_{n=1}^{\infty} \frac{d(n)}{n^2} = \zeta(2)^2 = \frac{\pi^4}{36}.
\]

Це красива та елегантна формула, що пов’язує арифметичну функцію кількості дільників із дзета-функцією Рімана та число \(\pi\).

---

\textbf{Відповідь:}

\[
\boxed{\frac{\pi^4}{36}}.
\]

\bigskip

\noindent\textbf{Задача 14.} (8) Всі вершини правильного 2024-кутника пофарбовані в червоний, зелений і синій кольори. Доведіть, що знайдуться два рівні 100-кутники, вершини яких є вершинами цього 2024-кутника та пофарбовані в один і той самий колір.

\textbf{Ідея:}  
Правильний 2024-кутник має 2024 вершини, рівномірно розташовані по колу. Будь-яка множина з 100 послідовних вершин утворює 100-кутник, який є рівним будь-якому іншому 100-кутнику, утвореному аналогічним чином (можна повернути весь 2024-кутник, щоб співпадали). 

Маємо три кольори для кожної вершини. Розглянемо всі можливі 100-кутники, утворені вибором 100 послідовних вершин на колі. Згідно з принципом Діріхле, принаймні два з цих 100-кутників матимуть однаковий кольоровий шаблон, оскільки кількість можливих кольорових шаблонів обмежена, а кількість 100-кутників — велика.

\textbf{Покрокове доведення:}

1. \textbf{Розглянемо 100-кутники, утворені послідовними вершинами:}  
   Позначимо вершини 2024-кутника як \(V_0, V_1, \dots, V_{2023}\) у часовому напрямку.  

   Для кожного \(i = 0, 1, \dots, 2023\) розглянемо 100-кутник, утворений вершинами:
   \[
   V_i, V_{i+1}, V_{i+2}, \dots, V_{i+99} \ (\text{індекси за модулем }2024).
   \]

   Таким чином, є 2024 таких 100-кутників, кожен визначений 100 вершинами, що утворюють дугу 2024-кутника.

2. \textbf{Рівність через поворот:}  
   Оскільки 2024-кутник є правильним (усі сторони й кути рівні), будь-які 100 послідовних вершин утворюють 100-кутник, який є рівним будь-якому іншому 100-кутнику з 100 послідовних вершин. Достатньо просто повернути весь 2024-кутник.

3. \textbf{Кольорові шаблони та принцип Діріхле:}  
   Кожна вершина пофарбована в червоний, зелений або синій колір. Множина з 100 послідовних вершин утворює послідовність із 100 кольорів, обраних із множини \(\{R, G, B\}\).

   - Загальна кількість можливих кольорових шаблонів для 100-кутника становить \(3^{100}\).
   - Ми маємо 2024 шаблони (один для кожного 100-кутника, починаючи з кожної вершини).

   Хоча \(3^{100}\) — дуже велике число, нас цікавить те, що з-поміж 2024 послідовностей кольорів, що утворюються на колі, принаймні дві обов’язково будуть однаковими.

4. \textbf{Пошук однакових кольорових шаблонів:}  
   Розглянемо всі 2024 кольорові послідовності довжиною 100. Оскільки кількість можливих комбінацій обмежена, а кількість послідовностей — велика, гарантовано знайдуться дві ідентичні послідовності.

5. \textbf{Висновок:}  
   Як тільки знайдуться дві однакові кольорові послідовності, відповідні 100-кутники будуть рівними (за поворотом) і матимуть однаковий кольоровий шаблон на своїх вершинах. Таким чином, у 2024-кутнику існують два рівні 100-кутники, вершини яких пофарбовані однаково.

\textbf{Відповідь:}  
Так, існують такі два рівні 100-кутники з однаковим кольоровим шаблоном.


\bigskip

\noindent\textbf{Задача 15} (9) Нехай \( a_0, a_1, a_2, a_3, a_4, \dots, a_{100}, a_{101} \) — довільні натуральні числа, більші за 1. Нехай послідовності \( \{p_n\}_{n=1}^{101} \) та \( \{q_n\}_{n=1}^{101} \) визначені у наступний спосіб:
\[
p_1 = a_1, \quad p_{k+1} = a_k^{p_k}, \quad q_1 = a_{101}, \quad q_{k+1} = a_{k+1}^{q_k}, \quad \text{для кожного } k \in \{1, 2, \dots, 100\}.
\]

\textbf{(a) Доведіть, що останні 99 цифр десяткових записів чисел \( p_{101} \) та \( q_{101} \) однакові.}

\textbf{Інтуїція:}  
Ця задача зводиться до аналізу поведінки дуже великих степенів модуло \( 10^{99} \). Основна ідея полягає в тому, що будь-який результат піднесення до степеня в модульній арифметиці має періодичну структуру. Ми можемо використати теорему Ейлера або функцію Кармайкла, щоб обмежити та звести експоненти до менших чисел. Завдяки симетрії в побудові \( p_{101} \) та \( q_{101} \) їх залишки модуло \( 10^{99} \) співпадають, забезпечуючи однаковість останніх 99 цифр.

\textbf{Деталізований розв'язок:}

1. \textbf{Структура чисел:}  
   Обидві послідовності створюють величезні «вежі» степенів:
   \[
   p_{101} = a_{100}^{a_{99}^{\dots^{a_1}}}, \quad q_{101} = a_2^{a_3^{\dots^{a_{101}}}}.
   \]

2. \textbf{Модульна арифметика:}  
   Нас цікавить залишок від ділення чисел \( p_{101} \) та \( q_{101} \) на \( 10^{99} \). Завдяки теоремі Ейлера або функції Кармайкла ми знаємо, що піднесення до степеня модуло \( 10^{99} \) має повторювані значення. Це дозволяє обчислювати залишки для кожного рівня «вежі», зводячи їх до менших чисел.

3. \textbf{Симетрія в побудові:}  
   Зверніть увагу, що \( p_{101} \) починається з \( a_1 \), а \( q_{101} \) починається з \( a_{101} \), але обидва проходять через одні й ті самі числа \( a_i \) лише у зворотному порядку. Завдяки цій симетрії, залишки модуло \( 10^{99} \) після проходження всієї «вежі» співпадуть.

4. \textbf{Висновок для частини (a):}  
   Таким чином, залишки від ділення на \( 10^{99} \) збігаються:
   \[
   p_{101} \equiv q_{101} \pmod{10^{99}}.
   \]
   Це означає, що останні 99 цифр десяткових записів \( p_{101} \) та \( q_{101} \) завжди однакові.

\bigskip

\textbf{(b) Чи можна це гарантувати для останніх 100 цифр?}

\textbf{Інтуїція:}  
При переході від 99 до 100 цифр ситуація значно ускладнюється. Регулярність модульної арифметики для \( 10^{99} \) більше не працює автоматично для \( 10^{100} \). Структура експонентів та модульних залишків стає більш складною, і гарантувати однаковість останніх 100 цифр неможливо.

\textbf{Деталізований розв'язок:}

1. \textbf{Аналіз модуло \( 10^{100} \):}  
   У випадку \( 10^{99} \), регулярність забезпечувалася теоремою Кармайкла, яка дозволяла «стиснути» експоненти. Але для \( 10^{100} \) цієї регулярності недостатньо. У модульній арифметиці \( 10^{100} \) починають відігравати роль нові залишки та періодичності, які не узгоджуються з модулем \( 10^{99} \).

2. \textbf{Контрприклад:}  
   Можна підібрати значення \( a_i \), для яких залишки \( p_{101} \) та \( q_{101} \) збігаються на 99 цифр, але відрізняються на 100-й. Це пов'язано з тим, що залишки модуло \( 10^{100} \) не є просто розширенням модуло \( 10^{99} \).

3. \textbf{Висновок для частини (b):}  
   Ми не можемо гарантувати однаковість останніх 100 цифр \( p_{101} \) та \( q_{101} \).

\bigskip

\noindent\textbf{Фінальні відповіді:}  
(a) Так, останні 99 цифр завжди однакові.  
(b) Ні, останні 100 цифр гарантувати не можна.


\end{document}
